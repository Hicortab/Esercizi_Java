\documentclass{article}

\begin{document}

\section{Teoria}

\begin{itemize}
    \item Che cos'è un flusso (stream) in Java? Qual è la differenza tra flussi di byte e di caratteri?
    \item Quali classi Java vengono utilizzate per la lettura e la scrittura di file di testo?
    \item Come si crea un file in Java? Quali metodi sono disponibili per verificare l'esistenza di un file?
    \item Come si legge un file di testo riga per riga in Java? Quali classi e metodi si utilizzano?
    \item Come si scrive su un file di testo in Java? Quali classi e metodi si utilizzano?
    \item Cos'è il buffering in I/O? Quali vantaggi offre l'uso di buffer nella lettura e scrittura di file?
    \item Come si gestisce la chiusura automatica delle risorse in Java durante le operazioni di I/O?
    \item Quali sono le differenze tra le classi \texttt{File}, \texttt{FileReader}, \texttt{BufferedReader}, \texttt{FileWriter} e \texttt{BufferedWriter}?
    \item Cos'è il package \texttt{java.nio.file} e come si differenzia dal package \texttt{java.io} nella gestione dei file?
    \item Come si gestiscono le eccezioni comuni durante le operazioni di I/O, come \texttt{IOException} e \texttt{FileNotFoundException}?
\end{itemize}

\section{Gestione dei File}

\begin{enumerate}
    \item \textbf{Creazione di un file e scrittura di dati} \\
    Scrivi un programma che crea un file chiamato \texttt{esempio.txt} e vi scrive la stringa ``Ciao, mondo!'' al suo interno.

    \item \textbf{Lettura di un file riga per riga} \\
    Scrivi un programma che apre il file \texttt{esempio.txt} e stampa ogni riga del file sulla console.

    \item \textbf{Scrittura di più righe su un file} \\
    Scrivi un programma che scrive tre righe di testo nel file \texttt{esempio.txt}, sovrascrivendo il contenuto esistente.

    \item \textbf{Aggiunta di contenuto a un file esistente} \\
    Scrivi un programma che apre il file \texttt{esempio.txt} in modalità append e aggiunge una nuova riga di testo alla fine del file.

    \item \textbf{Lettura di un file utilizzando \texttt{Files.readAllLines}} \\
    Scrivi un programma che legge tutte le righe del file \texttt{esempio.txt} utilizzando il metodo \texttt{Files.readAllLines} e le stampa sulla console.

    \item \textbf{Verifica dell'esistenza di un file} \\
    Scrivi un programma che verifica se il file \texttt{esempio.txt} esiste e stampa un messaggio appropriato sulla console.

    \item \textbf{Copia di un file} \\
    Scrivi un programma che copia il contenuto di \texttt{esempio.txt} in un nuovo file chiamato \texttt{copia.txt}.

    \item \textbf{Eliminazione di un file} \\
    Scrivi un programma che elimina il file \texttt{copia.txt} se esiste.

    \item \textbf{Lettura di un file con \texttt{Scanner}} \\
    Scrivi un programma che legge il contenuto del file \texttt{esempio.txt} utilizzando la classe \texttt{Scanner} e lo stampa sulla console.

    \item \textbf{Gestione delle eccezioni durante le operazioni di I/O} \\
    Scrivi un programma che tenta di aprire il file \texttt{esempio.txt} e gestisce eventuali eccezioni \texttt{IOException} e \texttt{FileNotFoundException} con messaggi appropriati.
\end{enumerate}

\end{document}
