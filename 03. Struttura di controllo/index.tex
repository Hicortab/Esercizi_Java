\documentclass{article}

\begin{document}

\section{Teoria}

\begin{itemize}

    \item Perché dovremmo utilizzare lo switch rispetto alla struttura di controllo?
    \item E' possibile non aprire le parentesi graffe nelle strutture di controllo? In quale caso?
    \item Quando ci conviene usare l'operatore ternario e come si usa? Scrivi un codice
    \item Quale pezzo della struttura di controllo (if, else-if ed else) può esistere senza gli altri pezzi? 
    \item Quanti else-if può avere la struttura di controllo?

\end{itemize}

\section{Struttura di controllo}

\begin{enumerate}
    \item \textbf{Controllo pari o dispari} \\
    Chiedi all'utente un numero intero e stampa se è \textit{pari}, \textit{dispari} o \textit{zero}.
    
    \item \textbf{Verifica voto scolastico} \\
    Data una valutazione numerica (0--100), stampa: \\
    \quad - ``Insufficiente'' per $<60$ \\
    \quad - ``Sufficiente'' per $60$--$69$ \\
    \quad - ``Buono'' per $70$--$79$ \\
    \quad - ``Distinto'' per $80$--$89$ \\
    \quad - ``Ottimo'' per $90$--$100$ \\
    \quad - ``Valore non valido'' per valori fuori range
    
    \item \textbf{Calcolo sconto in base all'età} \\
    Dato l'età di una persona, stampa lo sconto applicato su un biglietto: \\
    \quad - sotto 12 anni: 50\% \\
    \quad - 12--17 anni: 25\% \\
    \quad - 18--64 anni: nessuno sconto \\
    \quad - 65 anni e oltre: 40\%
    
    \item \textbf{Classifica di gara} \\
    Dato il tempo in secondi, assegna la categoria: \\
    \quad - $<10$s: ``Campione'' \\
    \quad - $10$--$20$s: ``Esperto'' \\
    \quad - $20$--$30$s: ``Intermedio'' \\
    \quad - $>30$s: ``Principiante''
    
    \item \textbf{Controllo orario} \\
    Data un'ora (0--23), stampa: \\
    \quad - ``Mattina'' per 5--11 \\
    \quad - ``Pomeriggio'' per 12--17 \\
    \quad - ``Sera'' per 18--21 \\
    \quad - ``Notte'' per 22--4
    
    \item \textbf{Controllo password} \\
    Chiedi una password all'utente e confrontala con una password corretta memorizzata. Stampa: \\
    \quad - ``Accesso consentito'' se uguali \\
    \quad - ``Password errata'' altrimenti
    
    \item \textbf{Gestione Multa e Punti Patente} \\
    Un'auto ha una \textbf{cilindrata} e una \textbf{velocità} rilevata in un tratto di strada. La velocità massima consentita è di \textbf{100 km/h}.

    \begin{enumerate}
        \item Calcola una \textbf{multa base} in base alla velocità:
        \begin{itemize}
            \item Fino a 110 km/h (ATTENZIONE): multa di 50 euro.
            \item Tra 111 e 130 km/h: multa di 100 euro.
            \item Oltre 130 km/h: multa di 300 euro.
        \end{itemize}
        
        \item Applica un \textbf{incremento del 20\%} alla multa se la cilindrata dell'auto è superiore a 2500 cc.
        
        \item Assegna i \textbf{punti di decurtazione} dalla patente:
        \begin{itemize}
            \item Se la velocità è superiore a 130 km/h: -5 punti.
            \item Se la velocità è tra 111 e 130 km/h: -3 punti.
            \item Altrimenti: nessuna decurtazione.
        \end{itemize}
        
        \item Stampa la \textbf{multa finale} e i \textbf{punti decurtati}.
    \end{enumerate}

    \vspace{0.5cm}

    \item \textbf{Valutazione e Bonus Studente} \\

    Un alunno ha una \textbf{media finale} e un numero di \textbf{assenze}.

    \begin{enumerate}
        \item Stabilisci lo \textbf{stato dello studente} (\textit{Bocciato}, \textit{Rimandato}, \textit{Promosso}, \textit{Ottimo}) in base alla media finale.
        
        \item Se lo studente è stato valutato come \textit{Promosso}, applica un \textbf{bonus} di 100 euro se:
        \begin{itemize}
            \item La media finale è maggiore o uguale a 8.5
            \item Le assenze totali sono inferiori a 10
        \end{itemize}
        
        \item Stampa gli eventuali \textbf{bonus} e \textbf{lo stato dello studente}.
    \end{enumerate}

    \vspace{0.5cm}

    \item \textbf{Controllo Semaforo Dinamico con Pedone} \\

    Il semaforo ha tre stati possibili, inseriti come stringa: \texttt{"rosso"}, \texttt{"giallo"} e \texttt{"verde"}. In aggiunta, c'è una variabile booleana \texttt{is\_pedone} che indica la presenza di un pedone che vuole attraversare.

    \begin{itemize}
        \item Se il semaforo è \texttt{"verde"}, l'istruzione è \textbf{Vai}.
        \item Se il semaforo è \texttt{"giallo"}, l'istruzione è \textbf{Attenzione}.
        \item Se il semaforo è \texttt{"rosso"}, l'istruzione è \textbf{Fermati}.
        \item Se è presente un pedone (\texttt{is\_pedone = true}), la priorità va a lui e l'istruzione da mostrare al conducente deve essere: \textbf{Fermati (priorità al pedone)}.
        \item Qualsiasi altro valore inserito per il colore del semaforo è da considerarsi non valido: l'istruzione sarà \textbf{Colore non valido}.
    \end{itemize}

    \medskip

    \noindent
    \textbf{Obiettivo:} stampa l'istruzione finale da mostrare al conducente, tenendo conto della presenza del pedone come priorità assoluta.


\end{enumerate}

\section{Switch}

\begin{enumerate}
    \item \textbf{Giorno della settimana} \\
    Dato un numero da 1 a 7, stampa il nome del giorno della settimana (1 = Lunedì, 7 = Domenica). Se il numero non è valido, stampa ``Numero non valido''.
    
    \item \textbf{Categoria prodotto} \\
    Data una lettera (\texttt{char}) che indica la categoria di un prodotto (ad esempio \texttt{'A'}, \texttt{'B'}, \texttt{'C'}), stampa una descrizione della categoria. Se la lettera non è prevista, stampa ``Categoria sconosciuta''.
    
    \item \textbf{Calcolo operazione base} \\
    Dati due numeri e un operatore (\texttt{+}, \texttt{-}, \texttt{*}, \texttt{/}), esegui l'operazione e stampa il risultato. Se l'operatore non è valido, stampa ``Operatore non valido''.

\end{enumerate}

\end{document}