\documentclass{article}

\begin{document}

\section{Teoria}

\begin{itemize}
    \item Cos'è una classe in Java? A cosa serve?
    \item Cos'è un oggetto? Come si crea un oggetto a partire da una classe?
    \item Che cosa rappresenta il costruttore? È obbligatorio dichiararlo?
    \item Cosa significa incapsulamento? Come si applica in Java?
    \item A cosa servono le parole chiave \texttt{this} e \texttt{new}?
    \item Cos'è la visibilità di un membro (\texttt{private}, \texttt{public})? Quali sono le differenze?
    \item È possibile creare più oggetti da una stessa classe? Cosa condividono?
    \item Cosa sono i metodi getter e setter? Perché sono importanti?
    \item È possibile creare classi dentro altre classi? In quali casi si usa?
\end{itemize}

\section{Classi e Oggetti}

\begin{enumerate}
    \item \textbf{Classe Persona} \\
    Crea una classe \texttt{Persona} con attributi \texttt{nome}, \texttt{età} e un metodo per stampare le informazioni.

    \item \textbf{Classe Rettangolo} \\
    Crea una classe \texttt{Rettangolo} con attributi \texttt{base} e \texttt{altezza}. Aggiungi un metodo che calcoli l'area.

    \item \textbf{Classe Studente} \\
    Crea una classe \texttt{Studente} con attributi \texttt{nome}, \texttt{cognome}, \texttt{mediaVoti}. Implementa un metodo che stampi se lo studente è promosso (\texttt{mediaVoti} $\geq 6$).

    \item \textbf{Classe Libro} \\
    Crea una classe \texttt{Libro} con attributi \texttt{titolo}, \texttt{autore}, \texttt{numeroPagine}. Crea un metodo per restituire una descrizione del libro.

\end{enumerate}

\subsection*{Esercizio: Sistema di Gestione Utenti}
Crea una classe \textbf{Utente} con i seguenti attributi:
\begin{itemize}
    \item \textbf{username} (stringa)
    \item \textbf{password} (stringa)
    \item \textbf{email} (stringa)
    \item \textbf{loggato}
\end{itemize}
Implementa i seguenti metodi:
\begin{itemize}
    \item Un metodo \texttt{registra(String newUsername, String newPassword, String newEmail)} che imposta i dati dell'utente. Restituisci \texttt{true} se la registrazione ha successo (se i campi non sono vuoti), altrimenti \texttt{false}.
    \item Un metodo \texttt{accedi(String usernameInserito, String passwordInserita)} che verifica se le credenziali inserite corrispondono a quelle dell'utente. Se le credenziali sono corrette, imposta l'attributo \texttt{loggato} a \texttt{true} e restituisci \texttt{true}, altrimenti \texttt{false}.
    \item Un metodo \texttt{cambiaPassword(String vecchiaPassword, String nuovaPassword)} che permette di cambiare la password, ma solo se l'utente è \texttt{loggato} e la vecchia password inserita è corretta.
\end{itemize}

\end{document}
