\documentclass{article}

\begin{document}

\section{Teoria}

\begin{enumerate}

    \item Se inserisco un valore con la virgola in una variabile int cosa succede?
    
    \item Indicare esplicitamente QUANDO e PERCHE' utilizzare una variabile e quando una costante
    
    \item Valuta il seguente codice, qual è l'output atteso? Cerca di spiegare perché il programma EVENTUALMENTE
    si comporta in maniera diversa in questi 2 casi.
    
    \begin{verbatim}
        String primaVariabile = "2";
        String secondaVariabile = "1";

        System.out.println(primaVariabile + secondaVariabile);

        int numero1 = 2;
        int numero2 = 1;

        System.out.println(numero1 + numero2);

    \end{verbatim}

    \item In un programma, quando dovrei utilizzare il tipo int, float, double, String? Secondo quale esigenza?

    \item Qual è la differenza tra float e double?
    
    \item \textbf{Approfondimento}: Perché il seguente codice dovrebbe dare i seguenti output (ammesso che lo faccia)
    
    \begin{verbatim}
        
        // Primo codice

        float a = 1/3;
		
		System.out.println(a);

        Output: 0.0

        // Secondo codice

        float a = (float)1/3;

        System.out.println(a);

        Output: 0.33333334

    \end{verbatim}

    \item Parlami della notazione camelCase e sneak\_case. A cosa servono e perché utilizzare una o l'altra?
    Ci sono zone del codice dove è CONSIGLIABILE usare una piuttosto che l'altra?

    \item Le variabili possono contenere valori inifiniti o finiti?
    
    \item Per maggiore leggibilità posso dividere le cifre di una variabile secondo un preciso criterio? 
    Se sì come? Scrivi un codice che lo mostri e indica il relativo output

    \item Come inserisco un valore esadecimale o binario nelle variabili int? Crea un codice

    \item Esistono delle regole di nomenclatura per le costanti o seguono le regole delle variabili?
    
    \item Indicare nella maniera più chiara possibile le regole per dichiarare una variabile
    senza far sollevare errori dal compilatore.

    \item Qual è la differenza tra dichiarazione e assegnazione? Se può aiutarti scrivi del codice a supporto
    della tua risposta.

\end{enumerate}

\section{Variabili}

\begin{enumerate}
    
    \item Scrivi un programma che presi i lati di un rettangolo calcoli la sua area e il perimetro. Ti ricordi come si fa...?
    Chi è l'input di questo programma? E l'output?

    \item Dati 2 numeri stampa la somma, differenza, prodotto e divisione di quest'ultimi
    
    \item Crea un programma che converta i km in metri 
    
    \item Chiedi il nome e il cognome all'utente e stampa il nome completo su una sola riga.
    
    \item Chiedi all'utente il suo anno di nascita e calcola l'età attuale.
    
    \item Dichiara una variabile intera e una variabile DOUBLE, sommale e visualizza il risultato.
    
    \item Chiedi all'utente di inserire un numero di minuti e converti il valore in ore e minuti (es. 130 minuti → 2 ore e 10 minuti).

    \item Chiedi all'utente un numero e stampa il suo quadrato.
    
    \item Hai tre variabili: a = 5, b = 10, c = 0. Sposta i valori in modo che:

    \begin{itemize}

        \item c contenga il valore di a
        \item a contenga il valore di b
        \item b contenga il valore originale di c

    \end{itemize}

    \item Dichiara una variabile int somma = 0. Chiedi tre numeri all'utente e ogni volta somma 
    il numero alla variabile somma, riassegnandola. Stampa la somma finale.

    \item Dichiara una variabile DI UN TIPO CORRETTO per eseguire le seguenti operazioni 
    ottenendo un risultato NON TRONCATO e:

    \begin{itemize}
        \item incrementala di 1
        \item moltiplicala per 2
        \item sottrai 3
        \item dividila per 2
    \end{itemize}

\end{enumerate}

\section{Costanti}

\begin{enumerate}
    
    \item Dichiara una costante PI con valore 3.1416. Chiedi il raggio all'utente e calcola l'area del 
    cerchio (A = PI * r * r). Prova a riassegnare PI per vedere cosa succede (errore di compilazione).

    \item Imposta una costante SCONTO\_FISSO = 20. Dichiara una variabile prezzoIniziale, calcola e stampa 
    il prezzo scontato. Sottolinea che il valore dello sconto non può essere cambiato.

    \item Imposta una costante TASSO\_CAMBIO = 1.12. Chiedi all'utente quanti euro vuole convertire e stampa il corrispettivo in dollari.

\end{enumerate}

\end{document}