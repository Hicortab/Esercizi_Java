\documentclass{article}

\begin{document}

\section{Teoria}

\begin{itemize}
    \item Cos'è una funzione (o metodo) in Java?
    \item Posso inserire l'output di una funzione \texttt{void} in una variabile? 
    E invece posso farlo con una funzione \texttt{int}? Giustifica le risposte 
    \item Come si definisce una funzione in Java?
    \item Qual è lo scopo dei parametri in una funzione? Possono esserci più parametri?
    \item Cosa significa il termine ``overloading'' delle funzioni?
    \item È possibile chiamare una funzione da un'altra funzione?
    \item Qual è la differenza tra variabili locali e parametri?
    \item Cos'è il valore di ritorno di una funzione? Come si restituisce?
    \item È possibile usare una funzione prima di dichiararla in Java?
    \item È possibile definire una funzione all'interno di un'altra? Perché?
\end{itemize}

\section{Funzioni}

\begin{enumerate}
    \item \textbf{Somma di due numeri} \\
    Scrivi una funzione che prende due numeri interi come parametri e restituisce la loro somma.

    \item \textbf{Controllo pari/dispari} \\
    Scrivi una funzione che prende un numero intero come parametro e restituisce \texttt{true} se è pari, \texttt{false} altrimenti.

    \item \textbf{Fattoriale} \\
    Scrivi una funzione che calcola il fattoriale di un numero intero positivo.

    \item \textbf{Massimo tra due numeri} \\
    Scrivi una funzione che riceve due numeri interi e restituisce il maggiore tra i due.

    \item \textbf{Verifica palindromo} \\
    Scrivi una funzione che riceve una stringa e restituisce \texttt{true} se è palindroma, \texttt{false} altrimenti.

    \item \textbf{Area di un rettangolo} \\
    Scrivi una funzione che riceve base e altezza di un rettangolo e restituisce la sua area.

    \item \textbf{Stampa menu} \\
    Scrivi una funzione \texttt{void} che stampa a schermo un menu di opzioni.

    \item \textbf{Potenza} \\
    Scrivi una funzione che prende due numeri interi: base ed esponente, e calcola la potenza.

    \item \textbf{Conta vocali} \\
    Scrivi una funzione che riceve una stringa e restituisce il numero di vocali presenti.

    \item \textbf{Conversione temperatura} \\
    Scrivi una funzione che converte gradi Celsius in Fahrenheit usando la formula: \\
    \quad $F = C \cdot \frac{9}{5} + 32$
\end{enumerate}

\end{document}
