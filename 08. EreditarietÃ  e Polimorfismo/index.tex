\documentclass{article}

\begin{document}

\section{Teoria}

\begin{itemize}
    \item Che cos'è l'ereditarietà in Java? A cosa serve?
    \item Come si dichiara una classe che estende un'altra classe?
    \item Qual è la differenza tra classe base (superclasse) e classe derivata (sottoclasse)?
    \item Cos'è il metodo \texttt{super} e quando si usa?
    \item Che cos'è il polimorfismo in Java? Come si manifesta?
    \item Cos'è l'overriding di un metodo? Come si differenzia dall'overloading?
    \item È possibile chiamare un metodo della superclasse da una sottoclasse? Come?
    \item Cos'è il binding dinamico (late binding) nei metodi?
    \item Che ruolo hanno i metodi astratti e le classi astratte nell'ereditarietà?
    \item Cos'è un'interfaccia in Java e come differisce dall'ereditarietà delle classi?
\end{itemize}

\section{Ereditarietà e Polimorfismo}

\subsection*{Esercizio 1: Libreria Digitale con ereditarietà multipla indiretta}
\textbf{Descrizione:}
\begin{itemize}
    \item Progettare un sistema per una libreria digitale.
    \item Esiste una classe astratta \texttt{ElementoBiblioteca} con:
    \begin{itemize}
        \item Attributi: \texttt{titolo}, \texttt{anno\_pubblicazione}
        \item Metodo astratto: \texttt{descrizione()}
    \end{itemize}
    \item Da essa derivano:
    \begin{itemize}
        \item \texttt{Libro} (aggiunge autore, numero pagine)
        \item \texttt{Rivista} (aggiunge numero, mese)
    \end{itemize}
    \item Entrambe devono ereditare anche da un'interfaccia \texttt{Stampabile} (o classe astratta) con il metodo \texttt{stampa()}.
    \item Creare una classe \texttt{Ebook} che eredita da \texttt{Libro} e implementa anche \texttt{Stampabile}, ma con logica diversa dalla stampa cartacea.
    \item Implementare una funzione \texttt{stampa\_info(elemento: ElementoBiblioteca)} che mostri polimorficamente le informazioni, indipendentemente dal tipo concreto.
\end{itemize}

\textbf{Sfida extra:}  
Se un \texttt{Ebook} viene trattato come \texttt{ElementoBiblioteca}, assicurarsi che la chiamata a \texttt{stampa()} esegua la versione corretta.

\subsection*{Esercizio 2: Sistema di pagamento polimorfico con override parziale}
\textbf{Descrizione:}
\begin{itemize}
    \item Creare una gerarchia di classi per un sistema di pagamenti:
    \begin{itemize}
        \item \texttt{Pagamento} (classe astratta) con:
        \begin{itemize}
            \item Attributo \texttt{importo}
            \item Metodo astratto \texttt{esegui()}
            \item Metodo concreto \texttt{riepilogo()} che mostra l'importo e chiama polimorficamente un metodo \texttt{dettagli\_transazione()} (che può essere sovrascritto)
        \end{itemize}
        \item \texttt{PagamentoCarta} e \texttt{PagamentoBonifico} ereditano e implementano \texttt{esegui()} e \texttt{dettagli\_transazione()}.
        \item \texttt{PagamentoCriptovaluta} eredita da \texttt{Pagamento} ma non sovrascrive \texttt{dettagli\_transazione()} (usa quello di default se presente).
    \end{itemize}
\end{itemize}

\textbf{Sfida extra:}  
Nel \texttt{riepilogo()}, se una classe non ridefinisce \texttt{dettagli\_transazione()}, far sì che venga mostrato un messaggio generico.

\subsection*{Esercizio 3: Zoo con metodi sovrascritti e chiamate alla superclasse}
\textbf{Descrizione:}
\begin{itemize}
    \item Creare una gerarchia:
    \begin{itemize}
        \item \texttt{Animale} (con attributi \texttt{nome}, \texttt{specie}, metodo \texttt{verso()} e metodo \texttt{info()}).
        \item \texttt{Mammifero} e \texttt{Uccello} ereditano da \texttt{Animale}.
        \item \texttt{Pinguino} eredita da \texttt{Uccello} ma sovrascrive \texttt{verso()} e \texttt{info()}:
        \begin{itemize}
            \item In \texttt{info()} deve chiamare prima la versione della superclasse, poi aggiungere dettagli extra.
        \end{itemize}
    \end{itemize}
    \item Creare \texttt{Zoo} con una lista di \texttt{Animale} e un metodo \texttt{mostra\_tutti()} che chiami polimorficamente \texttt{info()} su ogni animale.
\end{itemize}

\textbf{Sfida extra:}  
Aggiungere una funzione che accetta un \texttt{Animale} e, se è un \texttt{Pinguino}, chiama un metodo aggiuntivo \texttt{nuota()} senza rompere l’uso polimorfico per gli altri animali.


\end{document}
