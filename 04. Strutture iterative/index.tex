\documentclass{article}

\begin{document}

\section{Teoria}

\begin{itemize}
    \item Qual è la differenza tra i cicli \texttt{for}, \texttt{while} e \texttt{do-while}?
    \item In quali casi è preferibile usare un ciclo \texttt{for} anziché un \texttt{while}?
    \item Il ciclo \texttt{do-while} può non eseguire mai il blocco di codice? Motiva la risposta.
    \item È possibile creare un ciclo infinito? In quali casi potrebbe servire?
    \item Cosa succede se non aggiorno la variabile di controllo all'interno del ciclo?
    \item Qual è la sintassi corretta di un ciclo \texttt{for} in Java?
    \item È possibile annidare cicli? Quali accorgimenti bisogna prendere?
    \item Quando conviene usare la parola chiave \texttt{break} all'interno di un ciclo?
    \item A cosa serve la parola chiave \texttt{continue} in un ciclo?
\end{itemize}

\section{Strutture Iterative}

\begin{enumerate}
    \item \textbf{Stampa numeri da 1 a 10} \\
    Usa un ciclo \texttt{for} per stampare i numeri da 1 a 10 inclusi.

    \item \textbf{Somma dei primi 100 numeri naturali} \\
    Calcola e stampa la somma dei numeri da 1 a 100 usando un ciclo.

    \item \textbf{Tabellina di un numero} \\
    Chiedi all'utente un numero intero e stampa la sua tabellina fino a 10.

    \item \textbf{Conto alla rovescia} \\
    Stampa un conto alla rovescia da 10 a 1 utilizzando un ciclo \texttt{while}.

    \item \textbf{Numeri pari da 0 a 50} \\
    Usa un ciclo \texttt{for} per stampare solo i numeri pari compresi tra 0 e 50.

    \item \textbf{Somma fino a zero} \\
    Chiedi numeri interi all'utente finché non inserisce 0. Alla fine, stampa la somma di tutti i numeri inseriti (escludendo lo 0).

    \item \textbf{Password corretta} \\
    Usa un ciclo \texttt{do-while} per continuare a chiedere la password finché non viene inserita quella corretta.

    \item \textbf{Fattoriale di un numero} \\
    Chiedi un numero intero positivo e calcola il suo fattoriale.

    \item \textbf{Conta cifre di un numero} \\
    Chiedi un numero intero positivo e conta quante cifre contiene (es. 1234 → 4 cifre).

    \item \textbf{Numeri divisibili per 3} \\
    Stampa tutti i numeri da 1 a 100 divisibili per 3, saltando quelli divisibili anche per 5 (\texttt{continue}).
\end{enumerate}

\end{document}
