\documentclass{article}

\begin{document}

\section{Teoria}

\begin{itemize}
    \item Che cos'è un'eccezione in Java? Qual è il suo scopo?
    \item Qual è la differenza tra errori e eccezioni?
    \item Quali sono le principali classi di eccezioni in Java?
    \item Che cos'è il blocco \texttt{try-catch}? Come funziona?
    \item Cosa significa lanciare (throw) un'eccezione? Come si fa?
    \item Che cos'è il blocco \texttt{finally}? Quando viene eseguito?
    \item Qual è la differenza tra eccezioni controllate (checked) e non controllate (unchecked)?
    \item Come si crea una propria eccezione personalizzata?
    \item Che ruolo ha la parola chiave \texttt{throws} nella dichiarazione di un metodo?
    \item Perché è importante gestire correttamente le eccezioni?
\end{itemize}

\section{Eccezioni e Gestione degli Errori}

\begin{enumerate}
    \item \textbf{Divisione per zero} \\
    Scrivi un programma che chiede due numeri interi e calcola la divisione. Gestisci l'eccezione di divisione per zero con un blocco \texttt{try-catch}.

    \item \textbf{Parsing numerico} \\
    Scrivi un programma che chiede una stringa e prova a convertirla in un numero intero. Gestisci l'eccezione \texttt{NumberFormatException}.

    \item \textbf{Accesso a array} \\
    Scrivi un programma che accede a un elemento di un array dato un indice inserito dall'utente. Gestisci l'eccezione \texttt{ArrayIndexOutOfBoundsException}.

    \item \textbf{Lancio di eccezione personalizzata} \\
    Crea una classe di eccezione personalizzata chiamata \texttt{ValoreNonValidoException}. Scrivi un metodo che lancia questa eccezione se un valore passato è negativo.

    \item \textbf{Blocco finally} \\
    Scrivi un programma con un blocco \texttt{try-catch-finally} che stampa un messaggio sia in caso di eccezione sia alla fine dell'esecuzione.

    \item \textbf{Propagazione eccezioni} \\
    Scrivi un metodo che lancia un'eccezione e un metodo chiamante che la gestisce con un blocco \texttt{try-catch}.

    \item \textbf{Multipli catch} \\
    Scrivi un esempio di codice che gestisce diverse eccezioni (\texttt{ArithmeticException}, \texttt{NullPointerException}) con più blocchi \texttt{catch}.

    \item \textbf{Eccezioni runtime} \\
    Spiega e mostra un esempio di eccezione non controllata (\texttt{NullPointerException}).

    \item \textbf{Throws in dichiarazione metodo} \\
    Scrivi un metodo che dichiara di lanciare un'eccezione con \texttt{throws} e mostra come chiamarlo da un altro metodo.

    \item \textbf{Try-with-resources} \\
    Spiega cos'è il try-with-resources e scrivi un esempio di utilizzo con un \texttt{BufferedReader} per leggere un file (puoi simulare il codice senza file esterno).
\end{enumerate}

\end{document}
