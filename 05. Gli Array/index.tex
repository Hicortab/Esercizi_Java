\documentclass{article}

\begin{document}

\section{Teoria}

\begin{itemize}
    \item Che cos'è un array in Java e perché si utilizza?
    \item Qual è la differenza tra un array monodimensionale e uno bidimensionale?
    \item Come si accede a un elemento specifico in un array monodimensionale? E in uno bidimensionale?
    \item Cosa succede se accedo a un indice fuori dal range dell'array?
    \item È possibile cambiare la dimensione di un array dopo la sua dichiarazione? Perché?
    \item Qual è il valore predefinito degli elementi in un array di interi appena creato?
    \item Come si può determinare la lunghezza di un array?
    \item Come si inizializza un array con valori noti al momento della dichiarazione?
    \item In un array bidimensionale, gli "array interni" devono avere tutti la stessa lunghezza? 
    \item Quando conviene utilizzare un array bidimensionale rispetto a uno monodimensionale?
\end{itemize}

\section{Array}

\begin{enumerate}
    \item \textbf{Stampa di un array} \\
    Crea un array di 5 interi inizializzato con valori a tua scelta e stampalo usando un ciclo \texttt{for}.

    \item \textbf{Somma degli elementi} \\
    Dato un array di numeri interi, calcola e stampa la somma di tutti gli elementi.

    \item \textbf{Valore massimo} \\
    Scrivi un programma che trova il valore massimo in un array di interi.

    \item \textbf{Conta numeri pari} \\
    Dato un array di interi, conta e stampa quanti sono i numeri pari.

    \item \textbf{Media dei valori} \\
    Calcola e stampa la media dei valori contenuti in un array di \texttt{double}.

    \item \textbf{Array inverso} \\
    Dato un array, stampa i suoi elementi in ordine inverso.

    \item \textbf{Controllo presenza valore} \\
    Chiedi all'utente un numero e verifica se è presente in un array.
\end{enumerate}

\section{Matrici}
    \begin{enumerate}
        \item \textbf{Matrice stampa} \\
        Crea una matrice 3x3 di numeri interi e stampala in formato tabellare. \\
        Esempio: \\
        1 2 3 \\
        4 5 6 \\ 
        7 8 9 \\

        \item \textbf{Somma per riga} \\
        Dato un array bidimensionale 3x3, calcola e stampa la somma degli elementi di ogni riga.

        \item \textbf{Elementi maggiori di un valore} \\
        Dato un array bidimensionale, stampa tutti gli elementi maggiori di un certo valore fissato.
        
        \item \textbf{Cornice} \\
        Dato un array bidimensionale, imposta a 0 i bordi della matrice. \\ \\
        Esempio (matrice originale): \\
        1 2 3 4 5 6 \\
        7 8 9 1 2 3 \\
        4 6 7 8 1 2 \\

        Matrice mutata: \\
        0 0 0 0 0 0 \\
        0 8 9 1 2 0 \\ 
        0 0 0 0 0 0 \\ 

        \item \textbf{Diagonale} \\
        Calcola la somma della diagonale maggiore di una matrice n x n. Stampa la matrice finale e la somma

        \item \textbf{Modifica matrice} \\
        Crea un'applicazione che data una matrice n x n chieda all'utente un valore x e un altro y.
        Il programma dovrà cercare nella matrice il valore x e sostituirlo col valore y. Stampa la matrice finale

    \end{enumerate}

\end{document}
