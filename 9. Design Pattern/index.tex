\documentclass{article}

\begin{document}

\section{Teoria}

\begin{itemize}
    \item Che cos'è un design pattern? Perché è importante usarli nello sviluppo software?
    \item Descrivi il pattern Singleton. Qual è il suo scopo principale?
    \item Quali sono i principali problemi che il Singleton risolve?
    \item Come si implementa correttamente un Singleton in Java? Quali sono le varianti comuni?
    \item Quali sono i potenziali problemi del Singleton in contesti multithreading e come si possono risolvere?
    \item Che cos'è un Iterator? Qual è il suo scopo nei pattern di progettazione?
    \item Come funziona il pattern Iterator? Quali metodi fondamentali contiene un Iterator?
    \item Quali sono i vantaggi dell'uso di un Iterator rispetto all'accesso diretto agli elementi di una collezione?
    \item In Java, quali interfacce standard definiscono l'Iterator?
    \item Puoi fare un esempio di utilizzo di un Iterator per scorrere una collezione?
\end{itemize}

\section{Esercizi sul Singleton}

\begin{enumerate}
    \item \textbf{Implementazione Singleton semplice} \\
    Scrivi una classe \texttt{Configurazione} che implementa il pattern Singleton. Assicurati che ci sia un solo oggetto istanziato e che l'accesso sia tramite un metodo statico.

    \item \textbf{Singleton thread-safe} \\
    Modifica la classe \texttt{Configurazione} per renderla thread-safe usando la sincronizzazione.

    \item \textbf{Singleton con inizializzazione lazy} \\
    Implementa un Singleton che istanzia l'oggetto solo al primo utilizzo (lazy initialization).

    \item \textbf{Test del Singleton} \\
    Scrivi un breve programma che prova a creare più istanze della classe Singleton e verifica che si tratti della stessa istanza.

    \item \textbf{Singleton e serializzazione} \\
    Spiega come garantire che un Singleton rimanga unico anche dopo la serializzazione e deserializzazione.
\end{enumerate}

\section{Esercizi sull'Iterator}

\begin{enumerate}
    \item \textbf{Uso di Iterator su ArrayList} \\
    Crea un \texttt{ArrayList} di interi e usa un Iterator per stampare tutti gli elementi.

    \item \textbf{Iterator su una struttura dati personalizzata} \\
    Definisci una semplice classe \texttt{ListaDiNomi} che contiene una lista interna di stringhe. Implementa un metodo che restituisce un Iterator per scorrere i nomi.

    \item \textbf{Rimozione di elementi con Iterator} \\
    Usa un Iterator su una collezione per rimuovere tutti gli elementi che soddisfano una certa condizione (ad esempio numeri pari).

    \item \textbf{Implementazione manuale di un Iterator} \\
    Crea una classe \texttt{Contatore} che contiene un array di numeri interi e implementa manualmente un Iterator per scorrere questi numeri.

    \item \textbf{Confronto tra for-each e Iterator} \\
    Scrivi un breve testo che spiega le differenze tra l'uso del ciclo for-each e l'uso esplicito di un Iterator. Quando è preferibile usare uno rispetto all'altro?
\end{enumerate}

\end{document}
