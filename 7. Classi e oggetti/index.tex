\documentclass{article}

\begin{document}

\section{Teoria}

\begin{itemize}
    \item Cos'è una classe in Java? A cosa serve?
    \item Cos'è un oggetto? Come si crea un oggetto a partire da una classe?
    \item Che cosa rappresenta il costruttore? È obbligatorio dichiararlo?
    \item Cosa significa incapsulamento? Come si applica in Java?
    \item A cosa servono le parole chiave \texttt{this} e \texttt{new}?
    \item Cos'è la visibilità di un membro (\texttt{private}, \texttt{public})? Quali sono le differenze?
    \item È possibile creare più oggetti da una stessa classe? Cosa condividono?
    \item Cosa sono i metodi getter e setter? Perché sono importanti?
    \item È possibile creare classi dentro altre classi? In quali casi si usa?
\end{itemize}

\section{Classi e Oggetti}

\begin{enumerate}
    \item \textbf{Classe Persona} \\
    Crea una classe \texttt{Persona} con attributi \texttt{nome}, \texttt{età} e un metodo per stampare le informazioni.

    \item \textbf{Classe Rettangolo} \\
    Crea una classe \texttt{Rettangolo} con attributi \texttt{base} e \texttt{altezza}. Aggiungi un metodo che calcoli l'area.

    \item \textbf{Classe Studente} \\
    Crea una classe \texttt{Studente} con attributi \texttt{nome}, \texttt{cognome}, \texttt{mediaVoti}. Implementa un metodo che stampi se lo studente è promosso (\texttt{mediaVoti} $\geq 6$).

    \item \textbf{Classe Libro} \\
    Crea una classe \texttt{Libro} con attributi \texttt{titolo}, \texttt{autore}, \texttt{numeroPagine}. Crea un metodo per restituire una descrizione del libro.

    \item \textbf{Classe Banca} \\
    Crea una classe \texttt{ContoBancario} con attributi \texttt{saldo} e metodi \texttt{deposito} e \texttt{prelievo}.

    \item \textbf{Classe Punto} \\
    Crea una classe \texttt{Punto} con attributi \texttt{x} e \texttt{y}. Implementa un metodo che calcoli la distanza da un altro punto.

    \item \textbf{Classe Automobile} \\
    Crea una classe \texttt{Automobile} con attributi \texttt{marca}, \texttt{modello}, \texttt{velocità}. Aggiungi un metodo \texttt{accelera} che aumenti la velocità.

    \item \textbf{Classe Prodotto} \\
    Crea una classe \texttt{Prodotto} con attributi \texttt{nome}, \texttt{prezzo}, \texttt{quantità}. Aggiungi un metodo che calcoli il prezzo totale in base alla quantità.

    \item \textbf{Classe Film} \\
    Crea una classe \texttt{Film} con attributi \texttt{titolo}, \texttt{genere}, \texttt{durata}. Crea un metodo che stampi un messaggio se il film dura più di 120 minuti.

    \item \textbf{Classe Temperatura} \\
    Crea una classe \texttt{Temperatura} con un attributo in gradi Celsius e metodi per convertire in Fahrenheit e Kelvin.
\end{enumerate}

\end{document}
