\documentclass{article}

\begin{document}

\section{Teoria}

\begin{itemize}

    \item Perché dovremmo utilizzare lo switch rispetto alla struttura di controllo?
    \item E' possibile non aprire le parentesi graffe nelle strutture di controllo? In quale caso?
    \item Quando ci conviene usare l'operatore ternario e come si usa? Scrivi un codice
    \item Quale pezzo della struttura di controllo (if, else-if ed else) può esistere senza gli altri pezzi? 
    \item Quanti else-if può avere la struttura di controllo?

\end{itemize}

\section{Struttura di controllo}

\begin{enumerate}
    \item \textbf{Controllo pari o dispari} \\
    Chiedi all'utente un numero intero e stampa se è \textit{pari}, \textit{dispari} o \textit{zero}.
    
    \item \textbf{Verifica voto scolastico} \\
    Data una valutazione numerica (0--100), stampa: \\
    \quad - ``Insufficiente'' per $<60$ \\
    \quad - ``Sufficiente'' per $60$--$69$ \\
    \quad - ``Buono'' per $70$--$79$ \\
    \quad - ``Distinto'' per $80$--$89$ \\
    \quad - ``Ottimo'' per $90$--$100$ \\
    \quad - ``Valore non valido'' per valori fuori range
    
    \item \textbf{Calcolo sconto in base all'età} \\
    Dato l'età di una persona, stampa lo sconto applicato su un biglietto: \\
    \quad - sotto 12 anni: 50\% \\
    \quad - 12--17 anni: 25\% \\
    \quad - 18--64 anni: nessuno sconto \\
    \quad - 65 anni e oltre: 40\%
    
    \item \textbf{Classifica di gara} \\
    Dato il tempo in secondi, assegna la categoria: \\
    \quad - $<10$s: ``Campione'' \\
    \quad - $10$--$20$s: ``Esperto'' \\
    \quad - $20$--$30$s: ``Intermedio'' \\
    \quad - $>30$s: ``Principiante''
    
    \item \textbf{Controllo orario} \\
    Data un'ora (0--23), stampa: \\
    \quad - ``Mattina'' per 5--11 \\
    \quad - ``Pomeriggio'' per 12--17 \\
    \quad - ``Sera'' per 18--21 \\
    \quad - ``Notte'' per 22--4
    
    \item \textbf{Controllo password} \\
    Chiedi una password all'utente e confrontala con una password corretta memorizzata. Stampa: \\
    \quad - ``Accesso consentito'' se uguali \\
    \quad - ``Password errata'' altrimenti
    
    \item \textbf{Calcolo tassa automobilistica} \\
    Data la cilindrata di un'auto, applica: \\
    \quad - fino a 1000 cc: tassa 100€ \\
    \quad - 1001--1600 cc: tassa 200€ \\
    \quad - 1601--2500 cc: tassa 300€ \\
    \quad - oltre 2500 cc: tassa 500€
    
    \item \textbf{Controllo temperatura} \\
    Dato un valore in gradi Celsius, stampa: \\
    \quad - ``Gelo'' per temperature $<0$ \\
    \quad - ``Freddo'' per 0--10 \\
    \quad - ``Temperato'' per 11--20 \\
    \quad - ``Caldo'' per 21--30 \\
    \quad - ``Molto caldo'' per $>30$
    
    \item \textbf{Valutazione anno scolastico} \\
    Data la media finale di un alunno, stampa: \\
    \quad - ``Bocciato'' per media $<6$ \\
    \quad - ``Rimandato'' per media 6--6.9 \\
    \quad - ``Promosso'' per media 7--8.9 \\
    \quad - ``Ottimo'' per media $\geq 9$
    
    \item \textbf{Controllo traffico} \\
    Dato il colore del semaforo (inserito come stringa: ``rosso'', ``giallo'', ``verde''), stampa: \\
    \quad - ``Fermati'' per rosso \\
    \quad - ``Attenzione'' per giallo \\
    \quad - ``Vai'' per verde \\
    \quad - ``Colore non valido'' per altro
\end{enumerate}

\section{Switch}

\begin{enumerate}
    \item \textbf{Giorno della settimana} \\
    Dato un numero da 1 a 7, stampa il nome del giorno della settimana (1 = Lunedì, 7 = Domenica). Se il numero non è valido, stampa ``Numero non valido''.
    
    \item \textbf{Mese dell'anno} \\
    Data una variabile intera da 1 a 12, stampa il nome del mese corrispondente. Se il numero è fuori range, stampa ``Mese non valido''.
    
    \item \textbf{Categoria prodotto} \\
    Data una lettera (\texttt{char}) che indica la categoria di un prodotto (ad esempio \texttt{'A'}, \texttt{'B'}, \texttt{'C'}), stampa una descrizione della categoria. Se la lettera non è prevista, stampa ``Categoria sconosciuta''.
    
    \item \textbf{Calcolo operazione base} \\
    Dati due numeri e un operatore (\texttt{+}, \texttt{-}, \texttt{*}, \texttt{/}), esegui l'operazione e stampa il risultato. Se l'operatore non è valido, stampa ``Operatore non valido''.
    
    \item \textbf{Tipo di veicolo} \\
    Data una sigla (\texttt{"car"}, \texttt{"bike"}, \texttt{"truck"}), stampa il tipo di veicolo corrispondente. Usa \texttt{switch} con stringhe.
    
    \item \textbf{Valutazione con lettere} \\
    Data una lettera da \texttt{'A'} a \texttt{'F'}, stampa la valutazione corrispondente: \\
    \quad A - Eccellente \\
    \quad B - Buono \\
    \quad C - Sufficiente \\
    \quad D - Insufficiente \\
    \quad E - Scarso \\
    \quad F - Fallito \\
    Se la lettera è diversa, stampa ``Valutazione non valida''.
    
    \item \textbf{Calcolo giorni mese} \\
    Dato il numero di un mese (1-12), stampa quanti giorni ha quel mese (considera febbraio 28 giorni, non bisestile).
    
    \item \textbf{Risposta automatica} \\
    Data una stringa che rappresenta una risposta (\texttt{"yes"}, \texttt{"no"}, \texttt{"maybe"}), stampa una risposta automatica. Per altri input, stampa ``Risposta non riconosciuta''.
\end{enumerate}

\end{document}