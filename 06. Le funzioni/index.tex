\documentclass{article}

\begin{document}

\section{Teoria}

\begin{itemize}
    \item Cos'è una funzione (o metodo) in Java?
    \item Posso inserire l'output di una funzione \texttt{void} in una variabile? 
    E invece posso farlo con una funzione \texttt{int}? Giustifica le risposte 
    \item Come si definisce una funzione in Java?
    \item Qual è lo scopo dei parametri in una funzione? Possono esserci più parametri?
    \item Cosa significa il termine ``overloading'' delle funzioni?
    \item È possibile chiamare una funzione da un'altra funzione?
    \item Qual è la differenza tra variabili locali e parametri?
    \item Cos'è il valore di ritorno di una funzione? Come si restituisce?
    \item È possibile usare una funzione prima di dichiararla in Java?
    \item È possibile definire una funzione all'interno di un'altra? Perché?
\end{itemize}

\section{Funzioni}

\begin{enumerate}
    \item \textbf{Somma di due numeri} \\
    Scrivi una funzione che prende due numeri interi come parametri e restituisce la loro somma.

    \item \textbf{Controllo pari/dispari} \\
    Scrivi una funzione che prende un numero intero come parametro e restituisce \texttt{true} se è pari, \texttt{false} altrimenti.

    \item \textbf{Fattoriale} \\
    Scrivi una funzione che calcola il fattoriale di un numero intero positivo.

    \item \textbf{Massimo tra due numeri} \\
    Scrivi una funzione che riceve due numeri interi e restituisce il maggiore tra i due.

    \item \textbf{Verifica palindromo} \\
    Scrivi una funzione che riceve una stringa e restituisce \texttt{true} se è palindroma, \texttt{false} altrimenti.

    \item \textbf{Area di un rettangolo} \\
    Scrivi una funzione che riceve base e altezza di un rettangolo e restituisce la sua area.

\end{enumerate}

    \section*{Esercizio: Gestione Spese Familiari}

    L'obiettivo è creare un programma per la gestione e l'analisi delle spese familiari, suddividendo la logica in tre funzioni distinte che collaborano tra loro.

    \subsection*{Funzioni da implementare}

    \subsubsection*{`calcolaMediaSpese(int[] spese)`}
    Questa funzione accetta un \textbf{array di numeri interi} e calcola la \textbf{media aritmetica} di tutte le spese.
    \begin{itemize}
        \item \textbf{Parametro:} \texttt{int[] spese} (l'array delle spese).
        \item \textbf{Valore di ritorno:} \texttt{double} (la media delle spese).
    \end{itemize}

    \subsubsection*{`trovaSpesaMaggiore(int[] spese)`}
    Questa funzione accetta lo stesso array di spese e individua il valore massimo al suo interno.
    \begin{itemize}
        \item \textbf{Parametro:} \texttt{int[] spese} (l'array delle spese).
        \item \textbf{Valore di ritorno:} \texttt{int} (il valore della spesa maggiore).
    \end{itemize}

    \subsubsection*{`analizzaBudget(double media, int spesaMaggiore, int budgetMassimo)`}
    Questa funzione analizza i valori passati e restituisce una stringa che indica lo stato del budget familiare.
    \begin{itemize}
        \item Se la \texttt{spesaMaggiore} è superiore a \texttt{budgetMassimo}, restituisce ``Attenzione! Una singola spesa ha superato il budget.''.
        \item Altrimenti, se la \texttt{media} è superiore a \texttt{budgetMassimo}, restituisce ``Attenzione! La media delle spese ha superato il budget. Considera di ridurre le uscite.''.
        \item In tutti gli altri casi, restituisce ``Ottima gestione delle spese!''.
    \end{itemize}
    \begin{itemize}
        \item \textbf{Parametri:} \texttt{double media}, \texttt{int spesaMaggiore}, \texttt{int budgetMassimo}.
        \item \textbf{Valore di ritorno:} \texttt{String} (il messaggio di analisi).
    \end{itemize}

    \subsection*{Implementazione richiesta}
    \begin{enumerate}
        \item Creare un array di esempio con delle spese.
        \item Definire una costante per il budget massimo (es. \texttt{int budgetMassimo = 500;}).
        \item Nel metodo \texttt{main}, chiamare le funzioni nell'ordine corretto per ottenere i valori necessari.
        \item Stampare a console il messaggio finale restituito dalla funzione \texttt{analizzaBudget}.
    \end{enumerate}

\end{document}
